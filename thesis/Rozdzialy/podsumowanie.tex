\chapter{Podsumowanie} %wnioski i podsumowanie

W niniejszej pracy przedstawiono sposób modelowania zwierciadeł DBR w celu uzyskania najmniejszej możliwej dyspersji opóźnienia grupowego przy pomocy algorytmów rojowych. Przestawiony został, w sposób klarowny, schemat działania algorytmów rojowych oraz udowodniono, że faktycznie ta metoda działa. Przedstawiono również sposób wykorzystania algorytmów rojowych w celu znalezienia pożądanych zwierciadeł DBR, jak i wyniki, które z tego sposobu wynikały.

Uzyskane wyniki wykazują w sposób jasny, że wykorzystana metoda działa i jest możliwe wygenerowanie zwierciadeł ze względnie dobrą dyspersją opóźnienia grupowego. Można zauważyć, że najlepsze wyniki udało się uzyskać w przypadku pierwszego podejścia polegającego na losowaniu grubości warstw. W tym wypadku otrzymano GDD na poziomie $-800\fs$ i współczynnik odbicia równy $1$ na całym badanym przedziale. Te wyniki, mimo, że są dobre, polegają na znanych już strukturach zwierciadeł, które trudno pozyskać. Jedynymi sposobami na uzyskanie takich struktur jest korzystanie z wcześniej wykonanych prac, bądź też wygenerowanie struktury korzystając z jednego z dwóch pozostałych podejść.

Najgorsze wyniki otrzymano zdecydowanie korzystając z podejścia polegającego na losowaniu dwóch parametrów. Tutaj, w odróżnieniu od wyników uzyskanych przy pomocy algorytmów genetycznych w \cite{dbr2}, algorytmy rojowe nie sprawdziły się i otrzymane wyniki są najgorsze z otrzymanych, szczególnie pod kątem wariacji GDD na danych zakresie długości fal.

Podsumowując, otrzymane wyniki odbiegają znacznie od wyników uzyskanych w innych pracach, jak chociażby praca \cite{dbr2}. Świadczy to o tym, że temat algorytmów rojowych do modelowania dyspersyjnych własności zwierciadeł DBR pozostaje otwarty i należy znaleźć sposób na znaczne poprawienia wyników. Być może problem polega na wykorzystaniu generatora liczb pseudo-losowych w celu znalezienia nowych rozwiązań z sąsiedztwa i należy opracować nową metodę ich generowania.  