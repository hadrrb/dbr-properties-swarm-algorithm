\section{Metoda macierzy przejścia} \label{macierz} %3.	Opis metody macierzy przejścia 

Pierwszym etapem jest wyznaczenie funkcji pozwalającej na określenie wartości współczynnika odbicia R jak i GDD dla dowolnego zwierciadła DBR. Posłużymy się metodą macierzy przejścia, która polega na wyznaczeniu macierzy przejścia wektora natężenia pola elektrycznego z jednej warstwy do drugiej, a następnie, na tej podstawie, macierzy przejścia dla całej struktury.

Wiązka światłą padająca na zwierciadło DBR jest, zgodnie z istotą korpuskularno-falową światła, falą elektromagnetyczną. Pole elektryczne w każdej z warstw zwierciadła DBR ma postać:
\begin{equation}
    E_m(z) = F_me^{-in_m \frac\omega c z} + B_me^{in_m \frac\omega c z},
\end{equation}

gdzie $F_m$ i $B_m$ odpowiadają fali poruszającej się w kierunku $z$ z kolejno zwrotem dodatnim i ujemnym. \\
Pochodna powyższej wartości ma postać: 

\begin{equation}
    \frac{\partial E_m}{\partial z} = -i n_m \frac\omega c(F_me^{-in_m \frac\omega cz} - B_me^{in_m \frac\omega cz}).
\end{equation}

Na rysunku \ref{fig:transmatr} przedstawiono schemat ilustrujący rozkład sił pola elektrycznego w zwierciadle. Przyjmijmy $F'_m = F_m e^{-i\phi_m}$ i $B'_m = B_m e^{i\phi_m}$, gdzie $\phi = n_m \frac\omega c d_m$.
\begin{figure}[H]
    \centering
    \begin{tikzpicture}
        \draw (3,5) -- (3,0) 
        (1.5,5.3) node {$n_0$}
        (4,5.3) node {$n_1$}
        (3,-0.3) node {$z_0$};
        \draw (5,5) -- (5,0) 
        (6.5,5.3) node {$n_2$}
        (5,-0.3) node {$z_1$};
        \draw (8,5) -- (8,0) 
        (9,5.3) node {$n_3$}
        (8,-0.3) node {$z_2$};
        \draw (10,5) -- (10,0)
        (11.5,2.5) node {...}
        (10,-0.3) node {$z_3$};
        \draw (13,5) -- (13,0)
        (13,-0.3) node {$z_N$}
        (14,5.3) node {$n_N$};
        %skladowe F
        \draw[->] (2.25,3.5) -- (3,3.5) node[above left] {$F_0$};
        \draw[->] (3,3.5) -- (3.75,3.5) node[above] {$F_1$};
        \draw[color = red, ->] (4.25,3.5) -- (5,3.5) node[above left] {$F'_1$};
        \draw[->] (5,3.5) -- (5.75,3.5) node[above] {$F_2$};
        \draw[color = red, ->] (7.25,3.5) -- (8,3.5) node[above left] {$F'_2$};
        \draw[->] (8,3.5) -- (8.75,3.5) node[above] {$F_3$};
        \draw[color = red, ->] (9.25,3.5) -- (10,3.5) node[above left] {$F'_3$};
        \draw[->] (10,3.5) -- (10.75,3.5) node[above] {$F_4$};
        \draw[color = red, ->] (12.25,3.5) -- (13,3.5) node[above left] {$F'_{N-1}$};
        \draw[->] (13,3.5) -- (13.75,3.5) node[above] {$F_N$};
        %skladowe B
        \draw[<-] (2.25,1.5) -- (3,1.5) node[above left] {$B_0$};
        \draw[<-] (3,1.5) -- (3.75,1.5) node[above] {$B_1$};
        \draw[color = red, <-] (4.25,1.5) -- (5,1.5) node[above left] {$B'_1$};
        \draw[<-] (5,1.5) -- (5.75,1.5) node[above] {$B_2$};
        \draw[color = red, <-] (7.25,1.5) -- (8,1.5) node[above left] {$B'_2$};
        \draw[<-] (8,1.5) -- (8.75,1.5) node[above] {$B_3$};
        \draw[color = red, <-] (9.25,1.5) -- (10,1.5) node[above left] {$B'_3$};
        \draw[<-] (10,1.5) -- (10.75,1.5) node[above] {$B_4$};
        \draw[color = red, <-] (12.25,1.5) -- (13,1.5) node[above left] {$B'_{N-1}$};
        \draw[<-] (13,1.5) -- (13.75,1.5) node[above] {$B_{N}$};
    \end{tikzpicture}
    \caption{Schemat zwierciadła DBR}
    \label{fig:transmatr}
\end{figure}

Warunek ciągłości dla poszczególnych warstw można zapisać następująco:
\begin{equation}
    \left\{
    \begin{array}{l}
        F'_m + B'_m = F_{m+1} + B_{m+1}, \\
        n_m(F'_m - B'_m) = n_{m+1}(F_{m+1} - B_{m+1}).
    \end{array}
    \right.
\end{equation}
Rozwiązując powyższy układ równań względem $F'_m$ i $B'_m$ otrzymujemy:

\begin{equation}
    \left[
    \begin{array}{c}
    F_{m+1} \\ B_{m+1}
    \end{array}
    \right] = \frac 12
    \left[
    \begin{array}{cc}
         (1 + \frac{n_m}{n_{m+1}}) &  (1 - \frac{n_m}{n_{m+1}})\\
         (1 - \frac{n_m}{n_{m+1}}) &  (1 + \frac{n_m}{n_{m+1}})
    \end{array}
    \right]
    \left[
    \begin{array}{c}
    F'_{m} \\ B'_{m}
    \end{array}
    \right].
\end{equation}
Korzystając z definicji $F'_m$ i $B'_m$ możemy zapisać:
\begin{equation}
    \left[
    \begin{array}{c}
    F_{m+1} \\ B_{m+1}
    \end{array}
    \right] = \frac 12
    \left[
    \begin{array}{cc}
         (1 + \frac{n_m}{n_{m+1}}) &  (1 - \frac{n_m}{n_{m+1}})\\
         (1 - \frac{n_m}{n_{m+1}}) &  (1 + \frac{n_m}{n_{m+1}})
    \end{array}
    \right]
    \left[
    \begin{array}{c}
    e^{-i\phi_m} \\ e^{i\phi_m}
    \end{array}
    \right]
    \left[
    \begin{array}{c}
    F_{m} \\ B_{m}
    \end{array}
    \right],
\end{equation}
co ostatecznie daje:
\begin{equation}
    \left[
    \begin{array}{c}
    F_{m+1} \\ B_{m+1}
    \end{array}
    \right] = \frac 12
    \left[
    \begin{array}{cc}
         (1 + \frac{n_m}{n_{m+1}})e^{-i\phi_m} &  (1 - \frac{n_m}{n_{m+1}})e^{i\phi_m}\\
         (1 - \frac{n_m}{n_{m+1}})e^{-i\phi_m} &  (1 + \frac{n_m}{n_{m+1}})e^{i\phi_m}
    \end{array}
    \right]
    \left[
    \begin{array}{c}
    F_{m} \\ B_{m}
    \end{array}
    \right].
\end{equation}
Macierz $    \left[
    \begin{array}{cc}
         (1 + \frac{n_m}{n_{m+1}})e^{-i\phi_m} &  (1 - \frac{n_m}{n_{m+1}})e^{i\phi_m}\\
         (1 - \frac{n_m}{n_{m+1}})e^{-i\phi_m} &  (1 + \frac{n_m}{n_{m+1}})e^{i\phi_m}
    \end{array}
    \right]$ nazywamy macierzą przejścia i notujemy $T_m$.
    
Dla lepszej czytelności zapiszmy $E_m =     \left[
    \begin{array}{c}
    F_{m} \\ B_{m}
    \end{array}
    \right]$.
Można zauważyć, że wartość natężenia pola elektrycznego dla danej warstwy zależy od wartości natężenia dla warstwy poprzedniej:
    
\begin{align*}
    &E_1 = T_0 E_0, \\
    &E_2 = T_1 T_0 E_0, \\
    &... \\
    &E_N = (T_N ... T_1 T_0) E_0.
\end{align*}

Iloczyn poszczególnych macierzy przejścia, tj. macierz $T = T_N ... T_1 T_0$, nazywamy macierzą przejścia całej struktury.
Zapiszmy macierz przejścia całej struktury w sposób następujący:
$T = \left[
\begin{array}{cc}
   t_1  & t_2 \\
   t_3  & t_4
\end{array}
\right]$.

Korzystając z tak obliczonej macierzy przejścia można obliczyć amplitudowy współczynnik odbicia, który definiuje stosunek amplitudy fali odbitej do fali padającej. Jako, że $t_3$ odpowiada fali odbitej, natomiast $t_4$ fali padającej, można otrzymać wzór $r=-\frac{t_3}{t_4}$,
%Amplitudowy współczynnik transmisji $t = t_1 - \frac{t_2t_3}{t_4}$
oraz, na tej podstawie, współczynnik odbicia $R = rr^*$.
%Współczynnik transmisji $T = \frac{n_N}{n_0}tt^*$

Grupowe opóźnienie GD (Group Delay) można zdefiniować jako pierwsza pochodna zmiany fazy $\phi$ po częstotliwości $\omega$. Wiemy, że zmianę fazy można zapisać w postaci:
\begin{equation}
    \phi = \arctan\left(\frac{\mathrm{Im}(r)}{\mathrm{Re}(r)}\right),
\end{equation}
stąd GD obliczamy korzystając ze wzoru:

\begin{equation}
    GD = \frac{d\phi}{d\omega} = \frac{\mathrm{Im}(r) \frac{d}{d\omega}\mathrm{Re}(r) - \mathrm{Re}(r) \frac{d}{d\omega}\mathrm{Im}(r)}{rr^*} ,
\end{equation}
natomiast GDD jest pochodną GD po częstotliwości $\omega$, tak jak to przedstawiono we wzorze poniżej:

\begin{equation}
    GDD = \frac{d}{d\omega} GD.
\end{equation}

Tak otrzymane wzory posłużyły do sporządzenia modelu zwierciadła DBR (w postaci klasy) umożliwiającego obliczenie GDD i R dla danego zwierciadła. %Aby dokonać obliczeń wystarczy stworzyć nowy obiekt klasy \textit{DBRMirror} podając dane w postaci współczynników załamania i grubości dla poszczególnych warstw zwierciadła. Stworzenie takiego obiektu skutkuje automatycznym dokonaniem obliczeń i od tego momentu wartości GDD i R są dostępne jako atrybuty tej klasy.