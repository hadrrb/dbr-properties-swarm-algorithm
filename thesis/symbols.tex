\chapter*{Symbole przyjęte w pracy}
\label{app:symbole}
\addcontentsline{toc}{chapter}{Symbole przyjęte w pracy}

Jeśli w tekście nie wykazano inaczej, stosowane symbole należy rozumieć jako:

\begin{itemize}
\item[$N$] - liczba pszczół zwiadowców
\item[$n$] - liczba rozwiązań
\item[$n_{best}$] - liczba najlepszych rozwiązań
\item[$n_{qual}$] - liczba rozwiązań dobrych
\item[$n_{left}$] - liczba rozwiązań pozostałych (nie zdefiniowanych jako dobre lub najlepsze)
\item[$w_{best}$] - współczynnik określający procent rozwiązań uważanych za najlepsze względem wszystkich rozwiązań
\item[$w_{qual}$] - współczynnik określający procent rozwiązań uważanych za dobre względem wszystkich rozwiązań
\item[$w_{better}$] - współczynnik określający ile więcej pszczół będzie przypisana do przeszukiwania sąsiedztwa rozwiązań najlepszych
\item[$d_{near}$] - wartość określająca odległość przeszukiwania od rozwiązania pierwotnego (sąsiedztwo)
\item[$d_{far}$] - wartość określająca odległość przeszukiwania od danych początkowych (globalny obszar przeszukiwania)
\item[GDD] - dyspersja grupowego opóźnienia (ang. Group Delay Dispersion)
\item[R] - współczynnik odbicia
\item[$c_R$] - waga funkcji celu związana z wartością R
\item[$c_{avGDD}$] - waga funkcji celu związana ze średnią wartością GDD
\item[$c_{devGDD}$] - waga funkcji celu związana z odchyleniem standardowym od średniej GDD
\item[$c_{ptpGDD}$] - waga funkcji celu związana z różnicą między ekstremami GDD
\item[$n_{warstw}$] - liczba warstw w zwierciadle
\end{itemize}